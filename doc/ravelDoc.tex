\documentclass{article}
\usepackage{html,graphics}
\title{Ravel User Guide}

% prevent having to place annoying mboxes
%begin{latexonly}
\renewcommand{\htmladdimg}{\mbox{}}
%end{latexonly}

\begin{document}
\maketitle
\section{Installing Ravel}

\begin{itemize} 

\item You will need to choose either the 32 bit or the 64 bit version
of the Ravel Addin installer, as appropriate for your version of
Excel. You can install both versions on your system, but the installer
will not allow you to install both in the same directory. Double click
on the .msi file, and follow the instructions.

\item Next you will need to enable the Addin in Excel. 
\begin{itemize}
\item Within Excel,select "Excel Options", found within the Office
home button. 
\item Select "Manage: Excel Add-ins", and press "Go...".
\item Select "Browse", and browse to where Ravel is installed (by
default \verb+C:\Program Files\Ravel+).
\item Select the 32 bit (Ravel\_Win32.xll) or 64 bit (Ravel\_x64.xll)
as appropriate for you version of Excel. If you select the wrong
version, you will receive a message about it not being a valid Add-in.
\item Make sure the Ravel Add-in is enabled (ticked)
\item If the wrong version of Ravel is loaded, eg due to an ealier
Ravel install into a different place, the only way to correct
the situation permanently is to move or delete the incorrect XLL file.
\end{itemize}

\item Finally, you should create a "Quick Access" menu item for
creating a Ravel control of the active spreadsheet. To do this, select
the "Customize Quick Access Toolbar" menu, and select "More
commands...". Then choose the "newRavel" macro. This macro is also
available under the Add-Ins menu of the main toolbar.

\end{itemize}

\section{Getting started}

When getting started with Ravel, it is good to start with a
spreadsheet that is already formatted in a multidimensional way. The
following tables shows two 3D examples, \htmladdnormallink{one where all dimension labels are
stored in columns}{RavelEx1.xlsx}, and \htmladdnormallink{the other where one of the dimensions is spread
across the remaining columns}{RavelEx2.xlsx}:

\begin{tabular}{cc}
\begin{tabular}{|l|l|l|r|}
\hline
Year& Country& Gender& Sales\\\hline
1990& Australia& Male& 10400 \\
1990& Australia& Female & 12060\\
1990& UK & Male & 23012\\
1990& UK & Female& 22030\\
1991 & Australia & Male &12021\\
\multicolumn{4}{|c|}{$\cdots$}\\
\hline
\end{tabular} 
&
\begin{tabular}{|l|l|r|r|}
\hline
Gender & &Male& Female\\\hline
Year& Country& &\\\hline
1990& Australia& 10400 & 12060\\
1990& UK & 23012 & 22030\\
1991 & Australia & 12021 & 12900\\
\multicolumn{4}{|c|}{$\cdots$}\\
\hline
\end{tabular}
\\
\end{tabular}
In this case, the axes of the data set are labelled ``Year'',
``Country'' and ``Gender''.

These represent the two basic types of multidimensional spreadsheets
that Ravel will recognise automatically. However, Ravel provides tools
for handling more complex arrangements of multidimensional data, with
a modicum of user input, something we'll touch on later.

Load up one of the example dataset, and press the ``newRavel'' macro
button. A new workbook is created with two sheets, an input sheet
which is pretty much a copy of your original spreadsheet, and an
output sheet, which contains the ravelled result. As well as this, a
popup window is created, which allows you to manipulate the data. With
the above spreadsheet examples, the Ravel window looks like

\begin{center}
\htmladdimg{RavelEx.png}
\end{center}

Now click on the axis marked ``Country'' and drag it over the
``Gender'' axis. The axes switch positions, and the output spreadsheet
automatically updates its values to be country by year instead of
gender by year. You have just performed a rotation of the 3D datacube,
so that you are looking along the gender axis now. Similarly, notcie
the small spot, which was originally labelled ``Australia'', and is
now labelled ``Male''. This refers to the 2D slice of the datacube
that you are now viewing. Drag this spot to where it now says
``Female''. You have just changed the slice of the datacube to be
looking at female data. Are you interested in aggregating sales over
gender? Simply double click on the gender axis, and now the sales data
is aggregated over both genders. Double click again, and the axis
expands to allow you to examine the individual data items again.

Congratulations, you have now performed all the basic operations of
{\em OnLine Analytics Processing} (OLAP), namely {\em pivot} (or
rotation), {\em slice}, {\em roll-up} (aggregation, or reduction) and
{\em drill down} (reversing the aggregation operation). Ravel also
allows some forms of {\em dicing} by means the the {\em filter}
concept.

\section{Handling more difficult source spreadsheets}\label{difficult}


Consider what could happen if the first spreadsheet example had the
year axis in the 3rd column, instead of the first. It is very clear to
the human operator of the spreadsheet that Year is meant to be a third
axis of the datacube, as years and sales are not two items of the same
thing, but Ravel will instead assume that Year and Sales are separate
slices of some unnamed third dimension, because both columns contain
numerical data. You can force Ravel to have the correct interpretation
by pressing the "context button" (usually the right mouse button) on
the Year column, and selecting "Make column a Ravel axis".

More precisely speaking, "Make column a Ravel axis" indicates the
right-most column of the slice labels. Everything to the right of that
are data columns. Similarly, one can specify the rows which contain
slice label data, of which there may be more than one. For example

The text in the columns is coloured in blue to indicate the boundaries between
metadata (dimension names and slice labels, or coordinates), and the
actual data.


\begin{tabular}{|l|l|r|r|r|r|r|r|r|r|}
\hline
Year & & 1990 & & & & 1991 & & & \\\hline
Quarter& & Q1 & Q2 & Q3 & Q4 & Q1 & Q2 & Q3 & Q4 \\\hline
Country& Gender& &&&&&&&\\\hline
Australia& Male& 612 & 3060 & 2012 & 535 & 1036 & 833 & 2045 & 4013\\
Australia& Female & 734 & 1296 & 456 & 3456 & 220 & 2102 & 21 & 623\\
UK & Male & \multicolumn8{|c|}{$\cdots$}\\
UK & Female& \multicolumn8{|c|}{$\cdots$}\\
\multicolumn{4}{|c|}{$\cdots$}\\
\hline
\end{tabular} 

This example is available as \htmladdnormallink{RavelEx3.xlsx}{RavelEx3.xlsx}. When newRavel is run, a
separate axis is created for every quarter! If you open the input
sheet ``Sheet 1'', you will immediately see the problem --- all text
is colour blue, indicating that the automatic algorithm failed to
determine where the metadata stopped and the data started. By
selecting the context menu on column B and selecting ``Make column a
Ravel axis'', and then also selecting ``Make row a Ravel row on row 3,
you will see the meta data coloured blue, and the data black, and four
axes showing in the Ravel window (``Year'', ``Quarter'', ``Gender''
and ``Country'').

Ravel will normally treat rows with a single non-vacant cell as a
comment. However, you can also manually indicate that rows and columns are
comments using the ``Ignore row/col (treat as comment)'' context menu item.

\section{Merging/Splitting rows or columns}

In the example given in \S\ref{difficult}, you might want to merge the
quarter and year axes into a single axis (called quarter,
perhaps). Ravel provides the ability to merge multiple consecutive
rows or columns into a single axis descriptor. With RavelEx3.xlsx,
select the top two rows, and then select ``Merge'' from the row
context menu. The input spreadsheet will change to look like this:
\begin{tabular}{|l|l|r|r|r|r|r|r|r|r|}
\hline
Quarter& & 1990Q1 & 1990Q2 & 1990Q3 & 1990Q4 & 1990Q1 & 1990Q2 & 1990Q3 & 1990Q4 \\\hline
Country& Gender& &&&&&&&\\\hline
Australia& Male& 612 & 3060 & 2012 & 535 & 1036 & 833 & 2045 & 4013\\
Australia& Female & 734 & 1296 & 456 & 3456 & 220 & 2102 & 21 & 623\\
UK & Male & \multicolumn8{|c|}{$\cdots$}\\
UK & Female& \multicolumn8{|c|}{$\cdots$}\\
\multicolumn{4}{|c|}{$\cdots$}\\
\hline
\end{tabular} 
and the dataset has been transformed into a 3-dimension dataset. The
inverse operation can be achieved by selecting the row labelled
``Quarter'', and selecting ``Split'' from the context menu.

\section{Reductions}

In \S\ref{getting started}, you have already experienced summing up
data along an axis. In fact a range of {\em reduction} operations are
possible, including {\em sum}, {\em product} (multiplying), {\em
average} (summing then dividing by the number of slice labels along
that axis), {\em standard deviation}, and the {\em maximum} and {\em
minimum} elements.

You can select which reduction operation is applied to the rolled-up
axes from the ``Reduction" menu. You can also apply an individual
reduction operation to a single rolled-up axis without affecting any other axes
by means of the context menu attached to the reduction symbol attached to the
rolled-up axis.

\section{Filtering}

Filtering allows you to {\em dice} by selecting a range of the
datacube to in the output sheet. If you filter the {\em x} or {\em y} axes,
then a pair of calipers is drawn on the axis, allowing you to specify
the range of data you wish to consider.

If you filter by value, then a histogram control opens up on the right
of the Ravel control window. You may then drag the maximum and minimum
to control the maximum and minimum thresholds of data to be included
in the output sheet. This can be used to rapidly locate where
particular data values occur within the datacube.

\section{Sorting}

The {\em Sort} menu allows you to sort the data, either by the
slice labels of the {\em x} or {\em y} axes, or by the values of a particular
row or column. Rows and columns may be sorted independently of each
other.

When sorting by slice label, it is possible to specify whether
lexicographic sorting or numeric sorting is used. To understand the
difference, consider that the following sequence of numbers is
lexicographically sorted, not numerically sorted.

1 100 1000 11 200 2000

\section{Charts}

The chart menu is a convenience option for placing a chart on the
output sheet. By default, a surface chart is created, showing the
entire slice of data shown in the output sheet, unless that consists
of a single row or column.

You may further modify the chart properties using Excel's inbuilt
context menu.

\section{Persistence}

If you save the newly created Ravel workbook, when you open it up, the
Ravel will be in the same state as you left it.

Note that if you close the Ravel control window, the Ravel workbook is
also destroyed, unless you have made some edits to the input data sheet.

\section{Examples}

\htmladdnormallink{RavelEx1.xlsx}{RavelEx1.xlsx}\\
\htmladdnormallink{RavelEx2.xlsx}{RavelEx2.xlsx}\\
\htmladdnormallink{RavelEx3.xlsx}{RavelEx3.xlsx}

\end{document}
